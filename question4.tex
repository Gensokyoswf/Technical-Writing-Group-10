\documentclass{article}
\usepackage[utf8]{inputenc}


\begin{document}

\section{Review Result}

\subsection{Answer the research question}
question:  Previous methods has a crucial flaw.  Object models should not expose the representations of objects.
In case an one-to-many relationship for fileds represented as arrays, such as
an implementation of employees association with a vector:
\\
\\
\textbf{Vector employees;}
\\
\\This would result in an edge from Company to Vector , both
inappropriately exposing the representation of the
employees field and failing to show an association between
Company and Person.


To tackle the problem, Womble was an excellent tool[1].
With the following features, Womble does not suffer from this problem and
would still show an edge from Company to Person labeled
employees:
\begin{itemize}
    \item Associations are inferred by examining the methods
    of a class in addition to its declared fields. How a
    field is used provides more useful clues to its role
    than its declaration alone. Multiplicity and mutability of associations can both be determined.
    \item Container classes, such as Vector, are handled
    correctly most of the time: The container class does
    not appear as a node, but results in appropriate associations.
\end{itemize}


To extract UML from byte code, Womble makes use of most of the information in the
classfile. In particular, it uses the name of the class and the
names of classes it extends or implements, the declarations
of fields, static and virtual, the signatures of methods, and
the bytecode instructions within methods.


The basic structure of the object model is extracted as
follows: A node is introduced for each class, with a subset
edge for each extends or implements relationship. And,
roughly speaking, for each field declaration in a class A
of a field p of class B, an association labeled p from
A to B is added. Using its unique mechanism of inferring associations,
multiplicity and mutability.


 Womble uses a collection of simple heuristics to extract
object models from bytecode. The analysis is linear, mostly
monotonic (as classes are added), and reasonably accurate.
Despite its limitations,  it's an invaluable tool in our
everyday work.

\subsection{limitations}
Womble has several deficiencies. The major analysis flaw
is the inability to handle nested containers. It is quite
common, for example, for a class
C to have a field that is a
hashtable mapping strings to vectors of some other element
type E ; Womble fails to link C and E.
Womble also fails to
distinguish keys and values in a hashtable. Although
Womble's heuristics work well for many of the containers
that arise in practice, we have come to think that a more
systematic approach is needed.

\end{document}
